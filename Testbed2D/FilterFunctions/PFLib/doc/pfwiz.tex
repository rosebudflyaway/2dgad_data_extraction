\documentclass[12pt]{article}

\begin{document}

\section{Using {\tt PFWiz}}

The Particle Filter Toolbox in MATLAB, {\tt PFLib}, contains a
collection of Particle Filters implemented as MATLAB classes, together
with supporting functions such as those for resampling. One can write
MATLAB code using these classes by following {\tt PFLib}'s
API. Alternatively, one can use this {\tt PFWiz} to specify system and
parameter, and let {\tt PFWiz} automatically generate corresponding
MATLAB code.

Current version of {\tt PFLib} supports system models assumed by a
filter in the following form:
\begin{eqnarray*}
 x(k+1) &=& f(x(k)) + w(k), \\
y(k) &=& h(x(k)) + v(k),
\end{eqnarray*}
where $f(\cdot)$ and $h(\cdot)$ are nonlinear functions and $w$ and
$v$ are noises whose distributions are such that we can sample from
them and evaluate probability densities. These are to be specified in
the next screen of the GUI, followed by another screen to specify
filter choices. 

Simply click {\tt Next} to continue.

\section{Specifying system information}

In this screen you will be asked to specify the system model that your
filter assumes. (Note that this may be different from the system model that you
use for other purposes, for example, simulating the system trajectory.)

\begin{itemize}
\item {\bf dimensions: } The dimensions of the signal $x$ and
observation $y$.
\item {\bf initial condition: } The initial condition of the signal
$x$. Can be any valid MATLAB expression (that can be evaluated in the
current workspace). 
\item {\bf functions: } The signal transition function $f(\cdot)$ and
the observation function $h(\cdot)$. There are two ways to specify a
function, determined by the selection in the {\tt type} panel:
\begin{itemize}
\item {\bf script: } The function is stored in a MATLAB script. You
can type the function name (which should be on your current MATLAB path) into the edit
box, or you can click on the {\tt Browse} button to select the file.
\item {\bf anonymous:} The function is {\em not} stored in a file but
is an anonymous function whose expression you will type into the edit
box. As an example, suppose $x = [x_1, x_2]^T$, and the function
$f(\cdot)$ is
\[ f(x) = [\sin(x_1),  \cos(x_2)]^T, \]
then it can be specified as
\begin{verbatim}
	@(x) [sin(x(1));  cos(x(2))]
\end{verbatim}

\end{itemize}

\item {\bf noises: } Noise distribution can be selected from a
drop-down list of currently implemented ones. For Gaussian
distribution, {\tt mean} and {\tt covariance} boxes can have any valid
MATLAb expression (that can be evaluated in the current workspace),
e.g., {\tt eye(2)}. 

\end{itemize}

Click on the {\tt Show code} button, and one of two things happen:
\begin{itemize}
\item If there is no error, a MATLAb editor window will pop up to show
the code that can be generated so far. You do not have to save the
code right now, since you can do so in the next screen, after you choose your
filter.
\item An error message pops up. It usually tells you where the mistake
is. Sources of errors can be
\begin{itemize}
\item Unfilled field.
\item Invalid MATLAB expression.
\item Function specified not on MATLAB path.
\item Conflicting specifications, e.g., sizes of $x$ and $f(\cdot)$
mismatch. 
\item The {\tt Uniform} distribution has not been implemented yet;
don't choose it. 
\end{itemize}

\end{itemize}

The same error checking will also be done when you click on the {\tt
Next} button.

\section{Choosing a filter}

In this screen you will be asked to choose a filter of certain type,
and specify the necessary parameters. When a filter type is chosen
from a drop-down list, only fields that are relevant for its parameter
specification are shown.

These are the filter types that have been implemented:
\begin{itemize}
\item {\bf EKF:} Extended Kalman Filter. Parameters needed: Initial
distribution of $x$, and the
Jacobians $\partial f / \partial x$ and $\partial h / \partial
x$. These functions are specified either by a script name or as an anonymous
function. 
\item {\bf PF -- Simple: } Particles propagate according to the state
transition equation. Parameters needed: number of particles,
resampling period, resampling algorithm, initial distribution of the
particles. 
\item {\bf PF -- EKF: } Particles propagate according to a proposal
distribution. The proposal is obtained through an EKF-like procedure
which takes into account the current observation. Parameters needed in
addtion to those for {\tt PF -- Simple}: the Jacobian $\partial h /
\partial x$. Note that unlike in EKF, the Jacobian $\partial f /
\partial x$ is not needed, since the proposal is {\em conditioned on}
the previous value of $x$, as opposed to {\em propagated from} the
previous Gaussian approximation. 
\item {\bf PF -- Regularized: } Roughtly speaking, in representing the
filtered density, the delta function at each particle
location is replaced by a continuous kernel function. Parameters
needed in addition to those for {\tt PF -- Simple} are: 
\begin{itemize}
\item type: when regularization is performed. For details, see Project
Report.
\item whitening: whether to perform whitening on the particles. For
details, see Project Report.
\item width: kernel width, whose ``optimal'' value (as a function of
state dimension and number of particles) is filled into the edit box
by {\tt PFWiz}. It can be changed by a user.  For details, see Project
Report.
\end{itemize}
\item {\bf PF -- Auxilliary Variable: } Roughly speaking, this filter
approximates, using local Monte Carlo method, the ``optimal'' proposal
density that cannot be directly sampled from. Resampling is
automatically achieved in this implementation. Parameters needed:
Number of particles and their initial distribution. 
\end{itemize}

The following resampling schemes are available:
\begin{itemize}
\item {\bf none:} No resampling.
\item {\bf simple:} Resampling is performed using the weights.
\item {\bf residual: } Resampling is performed using the residual of
the weights, after the ``bulk part'' has been deterministically
chosen. 
\item {\bf systematic: } Only one random number is drawn, and
resampling is achieved by a set of points that are equally spaced from
this random number. 
\item {\bf branch-kill: } The resampling of each particle is
independent, therefore the total number of particles is not fixed. An
additional parameter, {\tt branch-kill threshold}, needs to be
specified: When the total number of particles drops below this
percentage, a simple resampling is performed to bring it back to  the
initially chosen value.
\end{itemize}

Resampling can be performed every $d$ step, where $d$ is specified in
the {\tt resampling period} box.

As in the previous screen, the {\tt Show code} button does error checking
and presents the code in an editor window. If the button {\tt Done} is
clicked, a confirmation window pops up to confirm the closing of {\tt
PFWiz}. The editor window will remain open showing the code
generated. 

\end{document}



